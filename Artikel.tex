\documentclass[a4paper,fleqn]{article}
\setlength{\headheight}{23pt}
\usepackage[a4paper,left=2.5cm,right=2.5cm,top=2.5cm,bottom=2.5cm]{geometry}
\usepackage[utf8]{inputenc}
%\usepackage[T1]{fontenc}
\usepackage[ngerman]{babel}
\usepackage{grffile}
\usepackage{eurosym}
\usepackage{graphicx} % Allows including images 
\usepackage{amsmath, amsthm, amssymb}
\usepackage{siunitx}
\usepackage{ifthen}
\usepackage{fancyhdr}
\pagestyle{fancy}
\usepackage[justification=raggedright,labelfont=bf,font=scriptsize]{caption}
\usepackage{todonotes}
\presetkeys{todonotes}{inline}{}
\usepackage{tikz}
\usetikzlibrary{calc}
\usepackage{lastpage}
\usepackage{xspace}	
\usepackage[skip=10pt plus1pt, indent=0pt]{parskip}
\usepackage{makecell}
\usepackage{csvsimple}
\usepackage{siunitx}
\usepackage{svg}
\usepackage{csquotes}
\usepackage{footmisc}
\usepackage{multicol}
\usepackage{tikz}
\usepackage{circuitikz}

%%%%%%%%%%%%%%%%%%%%%%%%%%%%%%%%%%%%%%%%%%%%%%%%%%%%%%%%%%%%%%%%%%%%%%%%%%%%%%%%
%%%%%%% VARIABLEN FÜR VORLAGE %%%%%%%%%%%%%%%%%%%%%%%%%%%%%%%%%%%%%%%%%%%%%%%%%%
%%%%%%%%%%%%%%%%%%%%%%%%%%%%%%%%%%%%%%%%%%%%%%%%%%%%%%%%%%%%%%%%%%%%%%%%%%%%%%%%
\newcommand\titlelab{Titel}
\newcommand\titlelecture{Vorlesung}
\newcommand\reportauthor{Autor}
%%%%%%%%%%%%%%%%%%%%%%%%%%%%%%%%%%%%%%%%%%%%%%%%%%%%%%%%%%%%%%%%%%%%%%%%%%%%%%%%
%%%%%%%%%%%%%%%%%%%%%%%%%%%%%%%%%%%%%%%%%%%%%%%%%%%%%%%%%%%%%%%%%%%%%%%%%%%%%%%%

%  Mathematik
\usepackage{amsmath}								%  Mathematikpaket
\allowdisplaybreaks									%  Formeln können auf mehrere Seiten verteilt werden
\usepackage{amssymb}								%  spezielle Mathmematiksymbole
\usepackage{siunitx}								%  korrektes Darstellen von Einheiten
\sisetup{locale = DE}

\usepackage{geometry}								%  Definition der Seitenraender
\usepackage[final]{pdfpages}							% zum einbinden von externen pdf-Seiten
\usepackage{graphicx}								%  zum Einbinden von Grafiken, spziell für dvi->ps Konvertierung
\setlength{\unitlength}{1mm}
\usepackage[hidelinks]{hyperref}	
\usepackage{placeins}								% FloatBarrier
\usepackage{upgreek}
\usepackage{tabularx}								% schöne tabellen
\usepackage{longtable}  								% lange tabellen
%\usepackage[osf,sc]{mathpazo}						% Schriftart	
\usepackage{booktabs}								% linien in der tabelle
\usepackage{nicefrac}								% brüche in zeilen
\usepackage{bold-extra}

\usepackage{enumitem}
\setitemize{itemsep=-5pt}
\setenumerate{itemsep=-5pt}
\setlength{\mathindent}{10pt}

\usepackage[					   % Literaturverzeichnis mit "biber [Projektname ohne Endung]" aktualisieren
    backend=biber,
    style=ieee,
]{biblatex}                        % Bibliography management
\addbibresource{references.bib}    % Bibliography file

\lhead{\titlelab} 
\chead{} 
\rhead{} 

\lfoot{} \cfoot{} \rfoot{\scriptsize Seite \thepage\ von \pageref{LastPage}} 
\renewcommand{\headrulewidth}{0.4pt}
\renewcommand{\footrulewidth}{0.4pt}
\renewcommand\familydefault\sfdefault	%Serifenlose Schriftart
\addto\captionsngerman{\renewcommand{\figurename}{Abbildung}}
\addto\captionsngerman{\renewcommand{\tablename}{Tabelle}}
%\captionsetup{justification = raggedright}

\newcommand\tab[1][0.5cm]{\hspace*{#1}}

\setlength{\belowrulesep}{1mm}

\begin{document}
\begin{center}
	\Large \textbf{\titlelab}
\end{center}
\tableofcontents
\hrulefill

\section{Abschnitt}
\begin{itemize}
    \item eine Aufzählung
    \item[1785] eine Aufzählung mit andrem Zeichen 
\end{itemize}

\subsection{Unterabschnitt}
\begin{figure}[!ht]
\centering
\resizebox{0.5\textwidth}{!}{%
\begin{circuitikz}
\tikzstyle{every node}=[font=\large]
\draw [ fill={rgb,255:red,0; green,0; blue,0} ] (9.5,13) circle (1.25cm);
\shade[ball color = black!80,opacity = 1] (9.5,13) circle (1.25cm);
\draw [line width=2pt, short] (11,13) -- (16,13);
\draw [line width=2pt, short] (3.75,13) -- (8,13);
\draw [line width=2pt, ->, >=Stealth] (4.25,13) -- (4.25,10.5);
\draw [line width=2pt, ->, >=Stealth] (5.25,13) -- (5.25,14);
\draw [line width=2pt, ->, >=Stealth] (6.5,13) -- (6.5,13.75);
\draw [line width=2pt, ->, >=Stealth] (12,13) -- (12,10.5);
\draw [line width=2pt, ->, >=Stealth] (13,13) -- (13,14);
\draw [line width=2pt, ->, >=Stealth] (14,13) -- (14,12);
\draw [line width=2pt, ->, >=Stealth] (15.25,13) -- (15.25,16.25);
\draw [line width=2pt, ->, >=Stealth] (3,16.75) -- (3,9.75);
\node [font=\large] at (3,9.5) {$+x$};
\node [font=\large] at (5.25,14.5) {$\vec F_A$};
\node [font=\large] at (6.5,14.25) {$\vec F_{R,f}$};
\node [font=\large] at (12,10) {$\vec F_g$};
\node [font=\large] at (13,14.5) {$\vec F_A$};
\node [font=\large] at (14,11.5) {$\vec F_{R,s}$};
\node [font=\large] at (15.25,16.5) {$\vec F_E$};
\node [font=\large] at (4.25,10) {$\vec F_g$};
\end{circuitikz}
}%

\label{fig:my_label}
\captionof{figure}{An einem Öltröpfchen angreifende Kräfte während des Fallens (links) und Steigens (rechts); $+x$ definiert die positive Richtung.\cite{oaverq4milver}}
\end{figure}
Ein Gleichugnsblock\footnote{Dabei können auch Fußnoten mit $V = \frac{4}{3}\pi \cdot r^3$ eingefügt werden.}
\begin{align}
	F_g &= m_{"Ol}\cdot g = \varrho_{"Ol}\cdot \frac{4}{3}\pi \cdot r^3 \cdot g \label{eq:F_g}\\
	F_A &= m_L \cdot g = \varrho_L\cdot \frac{4}{3}\pi \cdot r^3 \cdot g \label{eq:F_A}\\
	F_E &= q\cdot E = q \cdot \frac{U}{d} \label{eq:F_E}
\end{align}
Mehrere Spalten:
\begin{multicols}{2}
	\begin{tabbing}
		$\eta$ \tab \= – dyn. Viskosität der Luft \\
		$m_L$ \> – Masser der verdrängten Luft \\
		$\varrho_L$ \> – Dichte der Luft \\
		$m_{"Ol}$ \> – Masse des Öltröpfchens \\
		$\varrho_{"Ol}$ \> – Dichte des Öls \\
		$r$ \> – Radius des Öltröpfchens \\
		$v_f$ \> – Fallgeschwindigkeit \\
		$v_s$ \> – Steiggeschwindigkeit \\
		$q$ \> – Ladung des Öltröpfchens \\
		$E$ \> – elektrische Feldstärke im Kondensator \\
		$U$ \> – Kondensatorspannung \\
		$d$ \> – Abstand der Kondensatorplatten
	\end{tabbing}
\end{multicols}

\subsection*{unnumerierter Unterabschnitt}
\begin{tabular}{ll}
	Fallen: & $F_g - F_A - F_{R,f} = 0$ \\
	Steigen: & $F_g - F_A - F_E + F_{R,s} = 0$
\end{tabular} 
\newline

\subsubsection{Unterunterabschnitt}
Für Stichproben\footnotemark \space wird die Standardabweichung mit bekannter Grundgesamtheit $\sigma$ durch die Standardabweichung einer Stichprobe $s$ ersetzt und \textsc{Bessel}\footnotemark-korrigiert ($n \rightarrow n-1$)\cite{oA:2022:bk}:
\footnotetext{Eine Stichprobe liegt vor, da nur eine Auswahl der vorhandenen, nicht zählbaren Öltröpfchen vorliegt.}
\footnotetext{Die \textsc{Bessel}-Korrektur vergrößert die Varianz und ist damit realitätsnaher.}
\begin{align}
	s = \sqrt{\frac{1}{n-1}\cdot \sum _{i=1}^{n} \left(x_i-\mu\right)^2}
\end{align}

\newpage
\appendix
\section{Literaturverzeichnis}
\printbibliography
\end{document}